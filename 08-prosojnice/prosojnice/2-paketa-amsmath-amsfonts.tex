\begin{frame}{Matrike}
	% Naloga 3.2.1:
	% Oblikujte determinanto matrike. 
	% Vsebina matrike je že pripravljena v komentarju spodaj.
	Izračunajte determinanto
	$$
    \begin{vmatrix}
		-1 & 4 & 4 & -2 \\
		1 & 4 & 5 & -1 \\
		1 & 4 & -2 & 2 \\
		3 & 8 & 4 & 3 \\
	\end{vmatrix}
	$$
		

	V pomoč naj vam bo Overleaf dokumentacija o matrikah:
	
	\href{https://www.overleaf.com/learn/latex/Matrices}{\beamergotobutton{Matrices}}
\end{frame}

\begin{frame}{Okolje \texttt{align} in \texttt{align*}}
	% Naloga 3.2.2:
	% Okolje align je namenjeno poravnavi enačb.
	% Če ne želimo, da se enačba oštevilči, uporabimo okolje align*.
	% Nadomestite prikazni matematični način z okoljem align*.
	% Na ustreznih mestih vključite & in \\, da bo enačba videti kot v rešitvi.
	% Za pravilno postopno odkrivanje morate na enem mestu uporabiti ukaz `only',
	% ter na dveh mestih ukaz `onslide'.
	
	Dokaži \emph{binomsko formulo}: za vsaki realni števili $a$ in $b$ in za vsako naravno število $n$ velja
	\begin{align*}
	(a+b)^n \only<1>{&= \ldots} \\
	\onslide<2->{&= (a+b) (a+b) \dots (a+b)} \\
	\onslide<3->{&= a^n + n a^{n-1} b + \dots + \binom{n}{k} a^{n-k} b^k + \dots + n a b^{n-1} + b^n} \\
	&= \sum_{k=0}^n \binom{n}{k} a^{n-k} b^k
    \end{align*}
\end{frame}

\begin{frame}{Še ena uporaba okolja \texttt{align*}}
	% Naloga 3.2.3:
	% Oblikujte spodnje enačbe z okoljem align*.
	% Če naredite po en kurzor na začetku vsake vrstice, 
	% boste lahko oblikovali vse tri vrstice hkrati.
	
	Nariši grafe funkcij:
	\begin{align*}
	y &= x^2 - 3|x| + 2  &  y &= 3 \sin(\pi+x) - 2 \\
	y &= \log_2(x-2) + 3  & y &= 2 \sqrt{x^2+15} + 6 \\
	y &= 2^{x-3} + 1    &   y &= \cos(x-3) + \sin^2(x+1) 
	\end{align*}
\end{frame}

\begin{frame}{Okolje \texttt{multline}}
	% Naloga 3.2.4:
	% Oblikujte spodnje enačbe z ustreznim okoljem,
	% da bo enačba oblikovana kot v rešitvah.
	Poišči vse rešitve enačbe
	\begin{multline*}
	(1+x+x^2) \cdot (1+x+x^2+x^3+\ldots+x^9+x^{10}) = \\
	=(1+x+x^2+x^3+x^4+x^5+x^6)^2.
    \end{multline*}
\end{frame}

\begin{frame}{Okolje \texttt{cases}}
	% Naloga 3.2.5:
	% Oblikujte spodnji funkcijski predpis z ustreznim okoljem,
	% da oblikovan kot v rešitvah.
	Dana je funkcija
	$$
		f(x,y) = \begin{cases}
		    \frac{3x^2y-y^3}{x^2+y^2}; & (x,y) \ne (0,0), \\
			a; & (x,y) = (0,0).
	    \end{cases}
	$$
	\begin{itemize}
		% ukaz displaystyle preklopi v prikazni način v vrstici. 
	\item Določi $a$, tako da izračunaš limito \( \lim_{(x,y)\to(0,0)} f(x). \)
	\item Izračunaj parcialna odvoda $f_x(x,y)$ in $f_y(x,y)$.
	\end{itemize}
\end{frame}
