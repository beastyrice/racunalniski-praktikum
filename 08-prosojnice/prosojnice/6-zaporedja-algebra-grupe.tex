\begin{frame}{Zaporedja, vrste in limite}
	\begin{enumerate}
		\item 
		Naj bo $\sum_{n=1}^\infty a_n$ absolutno konvergentna vrsta in $a_n \ne -1$.
		Dokaži, da je tudi vrsta $\sum_{n=1}^\infty \frac{a_n}{1+a_n}$
		absolutno konvergentna.

		\item
		Izračunaj limito
		$$\lim_{x \to \infty} \left( \sin \sqrt{x+1} - \sin \sqrt{x} \right)$$

		\item
		Za dani zaporedji preveri, ali sta konvergentni.
		% Pomagajte si s spodnjima delno pripravljenima matematičnima izrazoma:
		$$ a_n = \underbrace{\sqrt{2+\sqrt{2+\dots+\sqrt{2}}}}_{n \text{ korenov}}  \qquad
		b_n = \underbrace{\sin(\sin(\dots(\sin 1)\dots))}_{n \text{ sinusov}}$$
	\end{enumerate}
\end{frame}

\begin{frame}{Algebra}
	\begin{enumerate}
		\item
		Vektorja $\vec{c} = \vec{a} + 2\vec{b} \text{ in } \vec{d} = \vec{a} - \vec{b}$
		sta pravokotna in imata dolžino 1. Določi kot med vektorjema $\vec{a}$ in $\vec{b}$.
		\item 
		Izračunaj
		$
        \begin{pmatrix}
        1 & 2 & 3 & 4 & 5 & 6 \\
        4 & 5 & 2 & 6 & 3 & 1
        \end{pmatrix}^{-2000}$
	\end{enumerate}
\end{frame}

\begin{frame}{Velika determinanta}
	Izračunaj naslednjo determinanto $2n \times 2n$, ki ima na neoznačenih mestih ničle.
	$$
    \begin{vmatrix}
    1  &    &       &       & 1      &       &       &  &  \\
       & 2  &       &       & 2      &       &       &  &  \\
       &    &\ddots &       & \vdots &       &       &  &  \\
       &    &       & n-1   & n-1    &       &       &  &  \\
    1  & 2  &\cdots & n-1   & n      & n+1   & n+2   & \cdots & 2n \\
       &    &       &      & n+1     & n+1   &       &  & \\
       &    &       &      & n+2     &       & n+2   &  &  \\
       &    &       &      & \vdots  &       &       & \ddots &  \\
	   &    &       &      &2n       &       &       &        & 2n
    \end{vmatrix}
$$
\end{frame}

\begin{frame}{Grupe}
	Naj bo
	\begin{align*}
    G &= \{ z \in \mathbb{C} : z = 2^k (\cos(m \pi \sqrt{2}) + i \sin(m \pi \sqrt{2})), \, k, m \in \mathbb{Z} \} \\
    H &= \{ (x, y) \in \mathbb{R}^2 : x, y \in \mathbb{Z} \}
	\end{align*}
	\begin{enumerate}
		\item
			Pokaži, da je $G$ podgrupa v grupi $(\mathbb{C} \setminus \{0\}, \cdot)$
			neničelnih kompleksnih števil za običajno množenje.
		\item
			Pokaži, da je $H$ podgrupa v aditivni grupi $(\mathbb{R}^2, +)$
			ravninskih vektorjev za običajno seštevanje po komponentah.
		\item
			Pokaži, da je preslikava $f:H\to G$, podana s pravilom
			$$(x,y) \mapsto 2^x(\text{cos}(y\pi\sqrt{2})+i\text{ sin}(y\pi\sqrt{2}))$$
			izomorfizem grup $G$ in $H$.
	\end{enumerate}
\end{frame}
